\documentclass{article}
% generated by Madoko, version 1.0.0-rc5
%mdk-data-line={1}


\usepackage[heading-base={2},section-num={False},bib-label={True}]{madoko2}


\begin{document}



%mdk-data-line={5}
\mdxtitleblockstart{}
%mdk-data-line={5}
\mdxtitle{\mdline{5}Optimization of the DNN program on the CPU+MIC Platform}%mdk
\mdxauthorstart{}
%mdk-data-line={10}
\mdxauthorname{\mdline{10}University of Electronic Secience and Technology of China}%mdk
\mdxauthorend\mdtitleauthorrunning{}{}\mdxtitleblockend%mdk

%mdk-data-line={7}
\section{\mdline{7}1.\hspace*{0.5em}\mdline{7}Analysis of the serial program}\label{sec-analysis-of-the-serial-program}%mdk%mdk

%mdk-data-line={9}
\noindent\mdline{9}First, we generate a call graph by using \mdline{9}\mdcode{Google~perfools}\mdline{9}, a open source performance profiler.sdfadf asadfasdfasdfasdf%mdk

%mdk-data-line={11}
\begin{itemize}[noitemsep,topsep=\mdcompacttopsep]%mdk

%mdk-data-line={11}
\item\mdline{11}Read the\mdline{11}~\href{http://research.microsoft.com/en-us/um/people/daan/madoko/doc/reference.html}{reference manual}\mdline{11}.%mdk

%mdk-data-line={12}
\item\mdline{12}Explore the upper-right toolbox menu to discover how Markdown works.%mdk

%mdk-data-line={13}
\item\mdline{13}\mdcode{Alt-Q}\mdline{13} reformats the current paragraph.%mdk
%mdk
\end{itemize}%mdk

%mdk-data-line={15}
\noindent\mdline{15}Enjoy!%mdk%mdk


\end{document}
